% PACKAGES

\usepackage{natbib}         % Pour la bibliographie
\usepackage{url}            % Pour citer les adresses web
\usepackage[T1]{fontenc}    % Encodage des accents
\usepackage[utf8]{inputenc} % Lui aussi
\usepackage[french]{babel}  % Pour la traduction française
\usepackage{numprint}       % Histoire que les chiffres soient bien
\usepackage{amsmath}        % La base pour les maths
\usepackage{mathrsfs}       % Quelques symboles supplémentaires
\usepackage{amssymb}        % encore des symboles.
\usepackage{amsfonts}       % Des fontes, eg pour \mathbb.
\usepackage{amsthm}
% \usepackage{lmodern}
\usepackage{anyfontsize}
\usepackage{cancel}
\usepackage{rotating}
\usepackage{appendixnumberbeamer}

%%% Pour L'utilisation de Python
\usepackage{minted}
% \usemintedstyle{monokai}

\usepackage{graphicx} % inclusion des graphiques
\usepackage{wrapfig}  % Dessins dans le texte.
\usepackage{stackengine}

\usepackage{tikz}     % Un package pour les dessins (utilisé pour l'environnement {code})
\usepackage[framemethod=TikZ]{mdframed}

% Algorithmes
\usepackage{algorithm}
\usepackage{frpseudocode}
\renewcommand{\listalgorithmname}{Table des algorithmes}
% Compléments pour transformer les algorithmes en francais
\algnewcommand\algorithmicnot{\textbf{ non }}
\algnewcommand\algorithmicand{\textbf{ et }}
\algnewcommand\algorithmicor{\textbf{ ou }}
\algrenewcommand\algorithmicrequire{\textbf{Entrée : }}
\algrenewcommand\algorithmicensure{\textbf{Sortie : }}


% COMMANDES

\newenvironment{code}{%
\begin{mdframed}[linecolor=white,innerrightmargin=0,innerleftmargin=0,
backgroundcolor=white,skipabove=0,skipbelow=0,roundcorner=5pt,
splitbottomskip=0,splittopskip=0]
}{%
\end{mdframed}
}

\newcommand\sources{\fontsize{4pt}{7.2}\selectfont}
\newcommand\codesize{\fontsize{6pt}{7.2}\selectfont}

\newcommand\R{\mathbb{R}}
\DeclareMathOperator{\Vor}{\mathrm{Vor}}

\newtheorem{defini}{Définition}
\theoremstyle{definition}
\newtheorem{thm}{Théorème}